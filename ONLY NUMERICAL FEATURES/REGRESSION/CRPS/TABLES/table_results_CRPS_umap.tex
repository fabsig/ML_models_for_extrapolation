% latex table generated in R 4.1.1 by xtable 1.8-4 package
% Fri Apr 26 15:03:14 2024
\begin{table}[ht!]
\centering
\begingroup\footnotesize
\begin{tabular}{lllllllllllll}
  \hline
\hline
task\_id & const. & lin. reg. & GAM & RF & GBT & GP & engression & MLP & ResNet & FT-Trans. & DRF & DGBT \\ 
  \hline
361072 & 19.5 & 9.98 & 6.46 & 16 & 14.6 & 13.1 & 11.4 & 3.32 & 13.8 & \textbf{3.25} & 15.9 & 15.8 \\ 
  361073 & 32.2 & 27.6 & 15.4 & 3.17 & 2.47 & 30.9 & \textbf{1.7} & 12.9 & 31 & 12.8 & 22 & 3.81 \\ 
  361074 & 3.40e-03 & 1.50e-03 & 1.29e-03 & 1.53e-03 & 1.26e-03 &  & \textbf{1.08e-03} & 2.37e-03 & 2.35e-03 & 2.68e-03 & 1.61e-03 & 1.88e-03 \\ 
  361076 & 0.488 & 0.413 & 0.436 & 0.412 & 0.434 & 0.497 & 0.392 & 0.432 & 0.518 & 0.42 & \textbf{0.379} & 0.425 \\ 
  361077 & 2.23e-04 & 9.56e-05 & 9.35e-05 & 9.11e-05 & \textbf{8.60e-05} & 1.22e-04 & \textbf{8.60e-05} & 1.47e-04 & 2.12e-04 & 2.32e-04 & 9.09e-05 & 1.19e-04 \\ 
  361078 & 0.406 & 0.245 & 0.374 & 0.183 & 0.17 & 0.251 & 0.187 & 0.24 & 0.268 & 0.235 & 0.182 & \textbf{0.167} \\ 
  361079 & 0.453 & 0.409 & 0.39 & 0.242 & 0.256 & 0.345 & 0.222 & 0.338 & 0.413 & 0.341 & 0.229 & \textbf{0.212} \\ 
  361080 & 0.734 & 0.138 & 0.133 & 0.133 & 0.132 & 0.161 & 0.133 & 0.35 & 0.483 & 0.338 & 0.135 & \textbf{0.131} \\ 
  361081 & 0.569 & 0.261 & 0.372 & 8.87e-02 & 5.98e-02 &  & \textbf{1.57e-02} & 0.387 & 0.313 & 0.254 & 9.20e-02 & 3.59e-02 \\ 
  361082 & 0.887 & 0.673 & 0.418 & 0.406 & 0.405 & 0.633 & 0.417 & 0.564 & 0.729 & 0.571 & \textbf{0.38} & 0.386 \\ 
  361083 & 0.295 & 0.257 & 0.246 & 0.21 & 0.211 &  & 0.228 & 0.27 & 0.301 & 0.29 & \textbf{0.153} & 0.21 \\ 
  361084 & 0.293 & 0.397 & 0.228 & 0.142 & 0.122 & 0.206 & 0.137 & 0.265 & 0.272 & 0.263 & 0.15 & \textbf{0.118} \\ 
  361085 & 2.49e-02 & 1.93e-02 & 3.27e-02 & 1.46e-02 & 1.37e-02 & 2.24e-02 & 1.13e-02 & 2.03e-02 & 2.01e-02 & 1.96e-02 & \textbf{1.08e-02} & 1.55e-02 \\ 
  361086 & 0.531 & 0.28 & \textbf{4.79e-02} & 7.62e-02 & 5.74e-02 &  & 4.82e-02 & 0.167 & 0.22 & 0.171 & 5.11e-02 & 6.09e-02 \\ 
  361087 & 0.283 & 0.152 & 9.24e-02 & 9.17e-02 & 8.41e-02 & 9.09e-02 & 8.13e-02 & 0.205 & 0.271 & 0.205 & 9.35e-02 & \textbf{8.05e-02} \\ 
  361088 & 0.879 & 0.493 & 0.37 & 0.388 & 0.393 & 0.406 & 0.392 & 0.621 & 0.686 & 0.601 & \textbf{0.349} & 0.354 \\ 
  361279 & 1.45e-02 & 1.48e-02 & 1.44e-02 & 1.38e-02 & 1.39e-02 & 1.73e-02 & 1.41e-02 & 1.41e-02 & 1.48e-02 & 1.46e-02 & \textbf{1.26e-02} & 1.39e-02 \\ 
  361280 & 1.65 & 1.17 & 1.13 & 1.13 & 1.14 & 1.38 & \textbf{1.06} & 1.34 & 1.41 & 1.39 & \textbf{1.06} & 1.13 \\ 
  361281 & 1.8 & 1.8 & 1.77 & 1.77 & 1.78 & 2.62 & 1.67 & 1.79 & 1.81 & 1.8 & \textbf{1.64} & 1.74 \\ 
   \hline
Avg. diff. & 4.70e+02 & 2.43e+02 & 2.02e+02 & 65 & 45.7 & 1.75e+02 & \textbf{19.9} & 2.30e+02 & 3.06e+02 & 1.92e+02 & 1.15e+02 & 47.3 \\ 
  Avg. acc. & 0.764 & 34 & 53 & 72.4 & 75.2 & 30.6 & \textbf{84.7} & 25.4 & 2.87 & 23.3 & 84.5 & 74.7 \\ 
  Avg. rank & 11.2 & 7.68 & 6.05 & 4.47 & 3.89 & 8.6 & \textbf{2.89} & 7.68 & 10.1 & 7.74 & 3.58 & 3.42 \\ 
   \hline
\hline
\end{tabular}
\endgroup
\caption{Average test CRPS. 
                  Best results are bold. 
                  'Avg. diff.' denotes the average relative difference in \% of a method compared to the best method.
                  'Avg. acc.' denotes the average normalized accuracy in \% of a method.
                  'Avg. rank' denotes the average rank of a method.} 
\label{TABLES/table_results_CRPS_umap}
\end{table}
